\section{Navegacion de robots autonomos}
\label{navegacion_de_robots}

Un robot móvil es aquella máquina automática que puede moverse en un determinado entorno. En función de la forma en la que se mueva se pueden dividir en diferentes categorías: robots rodantes (emplean ruedas para moverse), andantes (emplean dos o más extremidades para desplazarse), reptantes (se arrastran por el suelo), nadadores, voladores…\\

En el campo de la robótica móvil, se conoce como navegación al guiado de un robot que parte de un punto inicial, a través de un entorno en el que existen ciertos obstáculos a evitar, para alcanzar el punto final deseado. Para que se pueda cumplir con un cierto éxito esta tarea, es necesario poder resolver cuatro grandes problemas: Generar un mapa, localizar al robot en el, generar una trayectoria que una los puntos inicial y final y, por último, ser capaz de seguir esa trayectoria sin desviarse.\\

En primer lugar, se necesita proporcionar al robot un mapa del entorno que posea una precisión adecuada. Este puede haber sido realizado de antemano, o generado por el propio robot, empleando para ello los diversos sensores con los que cuenta, al mismo tiempo que avanza. Esta técnica es conocida como Simultaneous Localization And Mapping, o SLAM, que busca que el robot pueda ser capaz de generar el mapa de su entorno y, al mismo tiempo,  pueda localizarse con éxito dentro de ese mapa. De entre todas las técnicas empleadas para su solución, las más efectivas son las basadas en técnicas probabilísticas, que parten del teorema de Bayes, para poder tener en cuenta las distintas fuentes de incertidumbre que van apareciendo a lo largo del problema. Los principales algoritmos en este campo son el Filtro Extendido de Kalman, los mapas de ocupación de celdillas o la solución factorizada del filtro de Bayes.\\

A continuación, el robot debe ser capaz de localizarse dentro del mapa del entorno. Si para generar dicho mapa se empleó la técnica de SLAM, el punto anterior y este se resuelven de forma simultánea. Si el mapa fue generado de forma externa, debe emplearse un sistema que permita al robot conocer su posición. Por lo general, existen tres familias de algoritmos para esta tarea, aquellos basados en el concepto de probabilidad bayesiana, aquellos que plantean la localización del robot como la solución a un problema de optimización, y aquellos que mezclan conceptos de los dos tipos anteriores, conocidos como métodos híbridos.\\

El teorema de Bayes establece que la probabilidad que la probabilidad de que se dé un suceso aleatorio A conociendo que otro suceso B ya ha ocurrido depende de la distribución de probabilidad condicional de B si A ha ocurrido y de la distribución de probabilidad marginal de A. Aplicado a la localización de robots móviles, se calcula la probabilidad a posteriori asociada a una posible posición del robot dentro del espacio de soluciones del problema con los datos extraídos de la odometría y de los sensores y, a continuación, partiendo del paso anterior, se estima la posición futura del robot. Dentro de esta familia, destacan los métodos de localización de Markov y el filtro de Kalman, aunque este último no permite obtener la posición del robot, simplemente sirve para mantener localizado al robot durante el movimiento.\\

En cuanto a los métodos de optimización, trabajan con una función de coste que asigna a cada posible posición del robot un valor de coste. Una vez se tiene definida dicha función, se usa uno de los métodos de resolución de problemas de optimización, como pueden ser los algoritmos evolutivos, para tratar de encontrar la solución cuyo valor de coste sea mínimo. En bastantes casos, el coste es la medida que representa la diferencia que hay entre las medidas de los sensores del robot y de las medidas estimadas que se obtendrían si el robot se encontrase en esa posición.\\

Una vez hecho esto, el robot debe ser capaz de elegir una ruta libre de obstáculos para poder desplazarse desde el punto inicial al punto final. Para resolver este problema, se puede tener en cuenta las características de movimiento del robot a emplear o se puede buscar una solución basada simplemente en el mapa del entorno  y los puntos inicial y final. La planificación se puede plantear de dos maneras, de forma explícita, construyendo el espacio libre, aunque estó puede ser excesivamente costoso computacionalmente hablando, o muestreando el espacio libre, sin llegar a definirlo por completo, lo que es suficiente para encontrar u buen camino, usando un número suficiente de muestras.\\

Un planificador es completo si es capaz de encontrar un camino, en caso de que exista, en un tiempo finito. Los planificadores completos de resolución garantizan una solución si la resolución de la rejilla con la que se representa el espacio libre es lo suficientemente alta. Por último, los planificadores probabilísticos, como el método de Monte Carlo, ofrecen una mayor probabilidad de encontrar el camino cuantas mas iteraciones se realicen.\\

En general, hay cinco tipos de planificadores. Los planificadores basados en rejilla representan el espacio de configuraciones como una cuadrícula, donde cada punto de esta representa una posible configuración del robot. Para encontrar el espacio libtre, se comprobará si el camino entre dos celdas contiguas está despejado, y en caso afirmativo, se añadirá al conjunto de puntos por los que puede pasar el robot. Por último, se usa un método de resolución de grafos para encontrar el camino. Este método no es muy recomendable para entornos muy grandes o no estáticos, ya que la rejilla crece de forma exponencial con el tamaño del escenario, y en caso de que algo cambie, hay que volver a calcular el espacio lbre .\\

Los planificadores basados en intervalos son similares a los que se acaban de ver, solo que en este caso, el tamaño de las celdas no tiene por que ser uniforme. Estas celdas se dividen en dos subconjuntos, $X^-$ y $X^+$, donde $X^-$ es el espacio por el que el robot se puede mover sin problemas, mientras que $X^+$ es el espació máximo del cual no se puede salir. Al igual que antes, se usa un método de resolución de grafos para encontrar la trayectoria, que tiene que estar contenida en $X^+$ y ser factible en $X^-$. Dada su similitud con los planificadores basados en rejilla, también comparte uno de sus problemas, y es que este método tampoco es muy recomendable en entornos muy grandes.\\

Los planificadores geométricos representan tanto los obstáculos como el robot como polígonos. Después, con grafos de visibilidad o descomposición de celdas, se calcula el camino para que el robot pueda pasar entre los obstáculos.\\

Los planificadores basados en campos de potencial representan el entorno como un campo de potencial, donde el objetivo definido para el robot genera una fuerza de atracción, mientras que los obstáculos generán una fuerza de repulsión. Son algoritmos muy poco costosos desde el punto de vista computacional, pero por otra parte, pueden quedarse atrapados en mínimos locales, aunque se han desarrollado métodos dentro de esta familia que solucionan este problema, como es el caso del Fast Marching.\\

Por último, los planificadores basados en muestreo son bastante efectivos para entornos de muchas dimensiones, son fáciles de implementar y además, son planificadores completos, por lo que siempre encontrarán un camino, si este existe. Se distribuyen una serie de muestras por el entorno (existen diferentes aproximaciones para esto), asegurandose de que no caigan dentro de ningún obstáculo. Después, se unen aquellos puntos entre los que el robot puede moverse sin colisionar. Por último, se usa algún método como el $A^*$ o el Dijkstra para encontrar el camino. Con estos planificadores, el hecho de no encontrar un camino, no significa que no exista, ya que lo mismo si se aumenta el número de muestras se podría obtener una trayectoria. Dentro de esta familia se encuentran los algoritmos PRM, o Probabilistic Road Map o los Rapidly-expanding Random Trees (RRT).\\ 

Por último, el robot debe ser capaz de seguir la ruta que planificó anteriormente. Esto supone resolver la forma de adecuar dicha ruta a los parámetros que caracterizan el movimiento del robot (en caso de no haberlo hecho a la hora de panificar la ruta), además de poder mantener un seguimiento de su localización a lo largo del desplazamiento. Además, debe seguir recibiendo información del entorno a través de los sensores para poder reaccionar adecuadamente a la aparición de un obstáculo imprevisto.\\

En este trabajo se ha optado por una alternativa poco ortodoxa en cuanto al método para extraer el mapa y la localización del robot, ya que esto se consigue mediante una cámara externa que toma imágenes del entorno y las procesa para extraer los datos deseados. Para la planificación, se ha empleado un algoritmo basado en el muestreo, en concreto el PRM, y el seguimiento de la trayectoria se realiza de forma directa, reorientandose respecto al siguiente punto del camino cada vez que se avanza un poco.\\     