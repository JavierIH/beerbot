\chapter{Introducción}
\label{introducción}

El objetivo de este trabajo es aplicar los conocimientos adquiridos tanto en esta asignatura como a lo largo del Máster en el desarrollo de un caso práctico de estudio. Para ello, en esta memoria se presenta el proyecto Beerbot (Beer Extraction Experimental Robot), un sistema compuesto por un robot móbil y una cámara aérea situada sobre el escenario. La camará se encargará de tomar imagenes del entorno, estas se procesarán para obbtener las características deseadas y posteriormente un algoritmo de planificación de trayectorias generará un camino que permita al robot cumplir su misión evitando obstáculos y siguiendo la ruta más directa. La tarea programada es recoger una lata en algún punto del entorno, y devolverla al punto inicial.\\

Este trabajo se divide en los siguientes apartados. Después de esta breve introducción, se llevará a cabo un breve estudio acerca de la navegación autónoma para robots móviles acerca de la impresión 3D, las diferentes tecnologías existentes y del uso de técnicas de impresión 3D en la fabricación de estos robots.\\

En la siguiente sección se pasará a hablar del proyecto Beerbot en sí. Se dividirá en tres subapartados. En el primero se presentará el proyecto y los objetivos a conseguir, además de mostrar el entorno de pruebas empleado. El siguiente apartado tratará sobre el hardware empleado, es decir, se comentará el diseño y funcionamiento del robot y el modelo de cámara con el que se va a trabajar. Por último, el último subapartado tratará acerca del software desarrollado, que se puede clasificar en trés etapas. La primera de ellas consiste en el procesamiento de las imagenes tomadas por la cámara, de las que se extraerán los obstáculos presentes en el entorno, la posición y orientación del robot y la posición de la lata (al ser cilíndrica, la orientación de la lata es irrelevante). La segunda etapa es la planificación de la trayectoria. Se usará un planificador probabilístico para muestrear el espacio libre y un método de resolución de grafos para encontrar el camino. Por último, la tercera etapa consiste en el mecanismo de control del robot, que transformará los datos extraídos del planificador en comandos de velocidad para las ruedas, y que se encargará también de asegurar que el robot no se desviará de la ruta seleccionada.\\

Por último, se mostrarán imagenes del sistema en funcionamiento y se comentarán los resultados obtenidos, además de mostrar algunas posibles mejoras que se han considerado pero no se han podido llegar a implementar, antes de cerrar este trabajo con una conclusión final.\\
