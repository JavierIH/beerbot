\chapter{Conclusión}
\label{conclusion}

Con este trabajo se pretendía desarrollar una aplicación acerca de robots móviles. Después de haber repasado todos los aspectos del proyecto, la principal conclusión que se puede extraer de este es que los objetivos planteados se han cumplido en su totalidad. Se ha conseguido implementar un algoritmo de planificación que trabaja junto a un algoritmo de procesamiento de imágenes para poder adquirir la información del entorno.\\

El robot ha sido capaz de alcanzar el punto final y recoger la lata, evitando los obstáculos presentes en el entorno. Como todos los métodos de planificación basados en el muestreo, este sistema mejora su capacidad de encontrar una trayectoria si se incrementa el número de nodos. Tanto el Dijkstra como el $A^*$ son algoritmos completos, por lo que siempre van a encontrar un camino en caso de que exista. Esto se ha puesto a prueba y se ha confirmado experimentalmente. En cuanto al sistema de visión, se ha conseguido implementar un algoritmo que segmenta el entrono de forma bastante acertada, especialmente teniendo en cuenta que la iluminación deja bastante que desear. Por último, el diseño de robot ha sido totalmente adecuado para la tarea a resolver.\\

Como última conclusión, decir que se ha desarrollado un método robusto y funcional para conseguir mover el robot por un entorno con objetos estáticos. Como posibles mejoras, se podría mejorar el sistema de seguimiento de la trayectoria y el tiempo de procesamiento para poder aplicar este método a entornos dinámicos.\\